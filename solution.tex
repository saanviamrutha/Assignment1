\let\negmedspace\undefined
\let\negthickspace\undefined
\documentclass[journal,12pt,twocolumn]{IEEEtran}
\usepackage{gensymb}
\usepackage{amssymb}
\usepackage{polynom}
\usepackage{stfloats}
\usepackage[cmex10]{amsmath}
\usepackage{amsthm}
\usepackage[export]{adjustbox}
\usepackage{bm}
\usepackage{longtable}
\usepackage{enumitem}
\usepackage{mathtools}
 \usepackage{tikz}
\usepackage[breaklinks=true]{hyperref}
\usepackage{listings}
\usepackage{color}                                            %%
\usepackage{array}                                            %%
\usepackage{longtable}                                        %%
\usepackage{calc}                                             %%
\usepackage{multirow}                                         %%
\usepackage{hhline}                                           %%
\usepackage{ifthen}                                           %%
\usepackage{lscape}     
\usepackage{float}
 \usepackage{graphicx}
 \usepackage{hyperref}
\graphicspath{{/Users/saanviamrutha/Downloads/}}
 \DeclareMathOperator*{\Res}{Res}
 \renewcommand\thesection{\arabic{section}}
\renewcommand\thesubsection{\thesection.\arabic{subsection}}
\renewcommand\thesubsubsection{\thesubsection.\arabic{subsubsection}}
\renewcommand\thesectiondis{\arabic{section}}
\renewcommand\thesubsectiondis{\thesectiondis.\arabic{subsection}}
\renewcommand\thesubsubsectiondis{\thesubsectiondis.\arabic{subsubsection}}
\renewcommand\figurename{}
\hyphenation{op-tical net-works semi-conduc-tor}
\def\inputGnumericTable{}                                 %%
\lstset{
frame=single, 
breaklines=true,
columns=fullflexible
}
\setlength{\columnsep}{1 mm}
\begin{document}
\newtheorem{theorem}{Theorem}[section]
\newtheorem{problem}{Problem}
\newtheorem{proposition}{Proposition}[section]
\newtheorem{lemma}{Lemma}[section]
\newtheorem{corollary}[theorem]{Corollary}
\newtheorem{example}{Example}[section]
\newtheorem{definition}[problem]{Definition}
\newcommand{\BEQA}{\begin{eqnarray}}
\newcommand{\EEQA}{\end{eqnarray}}
\newcommand{\define}{\stackrel{\triangle}{=}}
\newcommand*\circled[1]{\tikz[baseline=(char.base)]{
    \node[shape=circle,draw,inner sep=2pt] (char) {#1};}}
\bibliographystyle{IEEEtran}
\providecommand{\mbf}{\mathbf}
\providecommand{\pr}[1]{\ensuremath{\Pr\left(#1\right)}}
\providecommand{\qfunc}[1]{\ensuremath{Q\left(#1\right)}}
\providecommand{\sbrak}[1]{\ensuremath{{}\left[#1\right]}}
\providecommand{\lsbrak}[1]{\ensuremath{{}\left[#1\right.}}
\providecommand{\rsbrak}[1]{\ensuremath{{}\left.#1\right]}}
\providecommand{\brak}[1]{\ensuremath{\left(#1\right)}}
\providecommand{\lbrak}[1]{\ensuremath{\left(#1\right.}}
\providecommand{\rbrak}[1]{\ensuremath{\left.#1\right)}}
\providecommand{\cbrak}[1]{\ensuremath{\left\{#1\right\}}}
\providecommand{\lcbrak}[1]{\ensuremath{\left\{#1\right.}}
\providecommand{\rcbrak}[1]{\ensuremath{\left.#1\right\}}}
\theoremstyle{remark}
\newtheorem{rem}{Remark}
\newcommand{\sgn}{\mathop{\mathrm{sgn}}}
\providecommand{\abs}[1]{\left\vert#1\right\vert}
\providecommand{\res}[1]{\Res\displaylimits_{#1}} 
\providecommand{\norm}[1]{\left\lVert#1\right\rVert}
%\providecommand{\norm}[1]{\lVert#1\rVert}
\providecommand{\mtx}[1]{\mathbf{#1}}
\providecommand{\mean}[1]{E\left[ #1 \right]}
\providecommand{\fourier}{\overset{\mathcal{F}}{ \rightleftharpoons}}
%\providecommand{\hilbert}{\overset{\mathcal{H}}{ \rightleftharpoons}}
\providecommand{\system}{\overset{\mathcal{H}}{ \longleftrightarrow}}
	%\newcommand{\solution}[2]{\textbf{Solution:}{#1}}
\newcommand{\solution}{\noindent \textbf{Solution: }}
\newcommand{\cosec}{\,\text{cosec}\,}
\providecommand{\dec}[2]{\ensuremath{\overset{#1}{\underset{#2}{\gtrless}}}}
\newcommand{\myvec}[1]{\ensuremath{\begin{pmatrix}#1\end{pmatrix}}}
\newcommand{\mydet}[1]{\ensuremath{\begin{vmatrix}#1\end{vmatrix}}}
\newcommand*{\permcomb}[4][0mu]{{{}^{#3}\mkern#1#2_{#4}}}
\newcommand*{\perm}[1][-3mu]{\permcomb[#1]{P}}
\newcommand*{\comb}[1][-1mu]{\permcomb[#1]{C}}
\makeatletter
\@addtoreset{figure}{problem}
\makeatother
\let\StandardTheFigure\thefigure
\let\vec\mathbf
\renewcommand{\thefigure}{\theproblem}
\def\putbox#1#2#3{\makebox[0in][l]{\makebox[#1][l]{}\raisebox{\baselineskip}[0in][0in]{\raisebox{#2}[0in][0in]{#3}}}}
     \def\rightbox#1{\makebox[0in][r]{#1}}
     \def\centbox#1{\makebox[0in]{#1}}
     \def\topbox#1{\raisebox{-\baselineskip}[0in][0in]{#1}}
     \def\midbox#1{\raisebox{-0.5\baselineskip}[0in][0in]{#1}}
\vspace{3cm}
\title{ Assignment-1\\Class 10 ICSE-2019}
\author{AI21BTECH11026\\SAANVI AMRUTHA
	\thanks{}
}
% make the title area
\maketitle
\newpage
\bigskip
\begin{enumerate}[label= ,ref= ]
\item \textbf{Question 7(a)}\\ 
In the given figure $AC$ is a tangent to the circle with centre $O$. If $\angle$$ADB=55$$^{\circ}$, find $x$ and $y$. Give reasons for your answers.\\
\begin{figure}[h]
\includegraphics[scale=.28]{Figure_1.png}
\label{Fig}
\end{figure}

\textbf{Solution:} \\
Given,\\
  \begin{align}
\angle BDA&=55^\circ \\ 
\angle OCA&=x^\circ\\
 \angle AOE&=y^\circ
 \end{align} 
As $AC$ is a tangent to the given circle, 
\begin{align} 
\angle OAC=\angle BAD=90^\circ
\end{align} 
\\
Angle Sum Property for $\triangle OAC$, 
        \begin{align}
           \angle OAC+\angle OCA+\angle AOC&=180^\circ \\
             90^\circ+x^\circ+y^\circ&=180^\circ\\
          x^\circ+y^\circ&=90^\circ
               \end{align} 
             
               
Angle Sum Property for $\triangle ABD$,    
\begin{align}
               \angle ABD+\angle BAD+\angle BDA&=180^\circ\\
               \angle ABD+90^\circ+55^\circ&=180^\circ\\
               \angle ABD&=35^\circ
 \end{align}  
 Let the radius of the circle be $r$.\\

 \begin {align} 
\implies OB=OE=r
\end{align} 
Then, $\triangle BOE$ is an isosceles triangle.\\
 \begin{align} 
 \implies \angle ABD=\angle BEO
 \end{align}  
\textbf{In a triangle, exterior angle is equal to sum of the two opposite interior angles.}\\
\begin{align}
&\implies &\angle AOE&=\angle ABD+\angle BEO\\
&\implies &y&=2(\angle ABD)\\
&\implies &y&=70^\circ\\
&\implies &x&=20^\circ
\end{align}  

 
 
                
\end{enumerate}
\end{document}