\let\negmedspace\undefined
\let\negthickspace\undefined
\RequirePackage{amsmath}
\documentclass[journal,12pt,twocolumn]{IEEEtran}
%
% \usepackage{setspace}
 \usepackage{gensymb}
 \usepackage{graphicx}
%\doublespacing
 \usepackage{polynom}
%\singlespacing
%\usepackage{silence}
%Disable all warnings issued by latex starting with "You have..."
%\usepackage{graphicx}
\usepackage{amssymb}
%\usepackage{relsize}

%\usepackage{amsthm}
%\interdisplaylinepenalty=2500
%\savesymbol{iint}
%\usepackage{txfonts}
%\restoresymbol{TXF}{iint}
%\usepackage{wasysym}
\usepackage{amsthm}
%\usepackage{pifont}
%\usepackage{iithtlc}
% \usepackage{mathrsfs}
% \usepackage{txfonts}
 \usepackage{stfloats}
% \usepackage{steinmetz}
 \usepackage{bm}
% \usepackage{cite}
% \usepackage{cases}
% \usepackage{subfig}
%\usepackage{xtab}
\usepackage{longtable}
%\usepackage{multirow}
%\usepackage{algorithm}
%\usepackage{algpseudocode}
\usepackage{enumitem}
 \usepackage{mathtools}
 \usepackage{tikz}
% \usepackage{circuitikz}
% \usepackage{verbatim}
\usepackage{tfrupee}
\usepackage[breaklinks=true]{hyperref}
%\usepackage{stmaryrd}
%\usepackage{tkz-euclide} % loads  TikZ and tkz-base
%\usetkzobj{all}
\usepackage{listings}
\usepackage[latin1]{inputenc}
    \usepackage{color}                                            %%
    \usepackage{array}                                            %%
    \usepackage{longtable}                                        %%
    \usepackage{calc}                                             %%
   \usepackage{multirow}                                         %%
    \usepackage{hhline}                                           %%
    \usepackage{ifthen}                                           %%
  %optionally (for landscape tables embedded in another document): %%
    \usepackage{lscape}     
% \usepackage{multicol}
% \usepackage{chngcntr}
%\usepackage{enumerate}
    \usepackage{amsmath}
%\usepackage{wasysym}
%\newcounter{MYtempeqncnt}
\graphicspath{{/Users/saanviamrutha/Desktop/}}
\DeclareMathOperator*{\Res}{Res}
\DeclareMathOperator*{\equals}{=}
%\renewcommand{\baselinestretch}{2}


% correct bad hyphenation here
\hyphenation{op-tical net-works semi-conduc-tor}
                                %%

\lstset{
%language=C,
frame=single, 
breaklines=true,
columns=fullflexible
}

\begin{document}

\newtheorem{theorem}{Theorem}[section]
\newtheorem{problem}{Problem}
\newtheorem{proposition}{Proposition}[section]
\newtheorem{lemma}{Lemma}[section]
\newtheorem{corollary}[theorem]{Corollary}
\newtheorem{example}{Example}[section]
\newtheorem{definition}[problem]{Definition}
%\newtheorem{thm}{Theorem}[section] 
%\newtheorem{defn}[thm]{Definition}
%\newtheorem{algorithm}{Algorithm}[section]
%\newtheorem{cor}{Corollary}
\newcommand{\BEQA}{\begin{eqnarray}}
\newcommand{\EEQA}{\end{eqnarray}}
\newcommand{\define}{\stackrel{\triangle}{=}}
\newcommand*\circled[1]{\tikz[baseline=(char.base)]{
    \node[shape=circle,draw,inner sep=2pt] (char) {#1};}}
\bibliographystyle{IEEEtran}
%\bibliographystyle{ieeetr}
\providecommand{\mbf}{\mathbf}
\providecommand{\pr}[1]{\ensuremath{\Pr\left(#1\right)}}
\providecommand{\qfunc}[1]{\ensuremath{Q\left(#1\right)}}
\providecommand{\sbrak}[1]{\ensuremath{{}\left[#1\right]}}
\providecommand{\lsbrak}[1]{\ensuremath{{}\left[#1\right.}}
\providecommand{\rsbrak}[1]{\ensuremath{{}\left.#1\right]}}
\providecommand{\brak}[1]{\ensuremath{\left(#1\right)}}
\providecommand{\lbrak}[1]{\ensuremath{\left(#1\right.}}
\providecommand{\rbrak}[1]{\ensuremath{\left.#1\right)}}
\providecommand{\cbrak}[1]{\ensuremath{\left\{#1\right\}}}
\providecommand{\lcbrak}[1]{\ensuremath{\left\{#1\right.}}
\providecommand{\rcbrak}[1]{\ensuremath{\left.#1\right\}}}
\theoremstyle{remark}
\newtheorem{rem}{Remark}
\newcommand{\sgn}{\mathop{\mathrm{sgn}}}
\providecommand{\abs}[1]{\left\vert#1\right\vert}
\providecommand{\res}[1]{\Res\displaylimits_{#1}} 
\providecommand{\norm}[1]{\left\lVert#1\right\rVert}
%\providecommand{\norm}[1]{\lVert#1\rVert}
\providecommand{\mtx}[1]{\mathbf{#1}}
\providecommand{\mean}[1]{E\left[ #1 \right]}
\providecommand{\fourier}{\overset{\mathcal{F}}{ \rightleftharpoons}}
%\providecommand{\hilbert}{\overset{\mathcal{H}}{ \rightleftharpoons}}
\providecommand{\system}{\overset{\mathcal{H}}{ \longleftrightarrow}}
	%\newcommand{\solution}[2]{\textbf{Solution:}{#1}}
\newcommand{\solution}{\noindent \textbf{Solution: }}
\newcommand{\cosec}{\,\text{cosec}\,}
\providecommand{\dec}[2]{\ensuremath{\overset{#1}{\underset{#2}{\gtrless}}}}
\newcommand{\myvec}[1]{\ensuremath{\begin{pmatrix}#1\end{pmatrix}}}
\newcommand{\mydet}[1]{\ensuremath{\begin{vmatrix}#1\end{vmatrix}}}

%\numberwithin{equation}{subsection}
%\numberwithin{problem}{section}
%\numberwithin{definition}{section}
\makeatletter
\@addtoreset{figure}{problem}
\makeatother
\let\StandardTheFigure\thefigure
\let\vec\mathbf
%\renewcommand{\thefigure}{\theproblem.\arabic{figure}}
\makeatletter 
\renewcommand{\thefigure}{S\@arabic\c@figure}
\makeatother
\renewcommand{\thefigure}{\theproblem}
%\setlist[enumerate,1]{before=\renewcommand\theequation{\theenumi.\arabic{equation}}
%\counterwithin{equation}{enumi}
%\renewcommand{\theequation}{\arabic{subsection}.\arabic{equation}}
\def\putbox#1#2#3{\makebox[0in][l]{\makebox[#1][l]{}\raisebox{\baselineskip}[0in][0in]{\raisebox{#2}[0in][0in]{#3}}}}
     \def\rightbox#1{\makebox[0in][r]{#1}}
     \def\centbox#1{\makebox[0in]{#1}}
     \def\topbox#1{\raisebox{-\baselineskip}[0in][0in]{#1}}
     \def\midbox#1{\raisebox{-0.5\baselineskip}[0in][0in]{#1}}
\vspace{3cm}
       
\title{AI1110 Assignment-1 \\ ICSE 10,2019} 
\author{Saanvi Amrutha\\AI21BTECH11026} 
    
\maketitle
\newpage
\bigskip

\textbf{Question 7(a)}\\ 
 \text In the given figure $AC$ is a tangent to the circle with centre $O$. If $\angle ADB=55^\circ$, find $x$ and $y$. Give reasons for your answers.\\

 \begin{figure}[h]
\includegraphics[width=\columnwidth]{Figure.png}
\end{figure}

\solution \\
Given,  
\begin{align}
                   \angle BDA&=55^\circ \\ 
                   \angle OCA&=x^\circ\\
                   \angle AOE&=y^\circ
                    \end{align} 
As $AC$ is a tangent to the given circle, 
\begin{equation} 
\angle OAC=\angle BAD=90^\circ
\end{equation} 
\\
Angle Sum Property for $\triangle OAC$, 
        \begin{align}
           \angle OAC+\angle OCA+\angle AOC&=180^\circ \\
             90^\circ+x^\circ+y^\circ&=180^\circ\\
          x^\circ+y^\circ&=90^\circ
               \end{align} 
             
               
Angle Sum Property for $\triangle ABD$,    
\begin{align}
               \angle ABD+\angle BAD+\angle BDA&=180^\circ\\
               \angle ABD+90^\circ+55^\circ&=180^\circ\\
               \angle ABD&=35^\circ
 \end{align}  
 Let the radius of the circle be $r$.\\

 \begin {equation} 
\implies OB=OE=r
\end{equation} 
Then, $\triangle BOE$ is an isosceles triangle.\\
 \begin{equation} 
 \implies \angle ABD=\angle BEO
 \end{equation}  
\textbf{In a triangle, exterior angle is equal to sum of the two opposite interior angles.}\\
\begin{align}
&\implies &\angle AOE&=\angle ABD+\angle BEO\\
&\implies &y&=2(\angle ABD)\\
&\implies &y&=70^\circ\\
&\implies &x&=20^\circ
\end{align}  
\\\\
\textbf{METHOD-2}\\
\textbf{Steps for construction:}\\
i) Draw a circle with centre at origin $\vec{O}$ and radius $r$.\\
      Assume $r=5$.
\begin{equation}
\vec{O}= \myvec{0\\0}\\
\end{equation}
ii) Consider points 
\begin{align}
\vec A&=\myvec{0\\-r}\\
\implies \vec A &= \myvec{0\\-5}\\
\vec B&=\myvec{0\\r}\\
\implies \vec B &= \myvec{0\\5}
\end{align}
Join $\vec A$,$\vec B$.\\
iii) Consider a point $\vec C$ where
\begin{equation}
\vec C=\myvec{AC\\-r}\\
\end{equation}
From the given figure in the question,
\begin{align}
\tan \angle AOC&=\frac{AC}{OA}\\
\tan y&=\frac{AC}{r}\\
AC&=r\tan y\\
AC&=5\tan 70^\circ\\
\implies \vec C&=\myvec{5\tan 70^\circ\\-5}
\end{align}
Join $\vec A$,$\vec C$.\\
Join $\vec O$,$\vec C$.\\
iv) 
Consider a point $D$ where,\\
\begin{equation}
\vec D=\myvec{AD\\-r}\\
\end{equation}
From the given figure,
\begin{align}
\tan \angle ABD&=\frac{AB}{AD}\\
\tan 35^\circ&=\frac{2r}{AD}\\
AD&=10\tan 35^\circ \\
\implies \vec D &=\myvec{10\tan35^\circ\\-5}
\end{align}
Join $\vec B$,$\vec D$.\\\\
Using the above steps of construction, generate the figure using python.\\\\
\textbf{Generated Figure:}
\begin{figure}[h]
    \centering
    \includegraphics[scale=0.28]{Figure1.png}
\end{figure}
\\\\\\\\\\
The input and output parameters required for drawing the figure are available in the below table.\\
\begin{table}[!h]
    \begin{tabular}{|c|c|c|} \hline
        \textbf{Variable} & \textbf{Value}    & \textbf{Input/Output}          \\ \hline
        $r$               & 5               & Input          \\ \hline
        $\angle AOC$ &$70^\circ$   &Input       \\ \hline    
        $\angle ABD$   &$35^\circ$    &Input      \\ \hline
        $\vec{O}$       & $\vec{0}$        & Input  \\\hline
        $\vec{A}$       &\myvec{0\\-5}& Input\\\hline
        $\vec{B}$       &  $\myvec{0\\5}$ & Input\\\hline
        $\vec{C}$       &\myvec{5\tan 70^\circ\\-5}& Output\\\hline
        $\vec{D}$       & $\myvec{10\tan35^\circ\\-5}$ & Output\\\hline
        $\vec{E}$       &\myvec{5\sin 70^\circ \\ 5\cos 70^\circ} &Output\\\hline
    \end{tabular}
\end{table}
 \end{document}